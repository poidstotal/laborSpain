\section{Data Source and Methodology}\label{sec:lab_method}

  \subsection{Measuring the Aggregate Labor Supply}
  Aggregate labor supply \(L\) is estimated by the weighted sum of hours worked during the year by age \(a\) and by residence status \(r\) (\(whx_{ar}\)), with \(r = (\text{immigrant, native})\). 

  The weighting factor is the product of the total population size (\(pop_{r}\)), the population structure (\(str_{ar}\)), and the labor participation rates (\(wpx_{ar}\)) by age and residence status.

  \begin{equation}\label{eq:pclab}
    L = \displaystyle\sum_{r}\displaystyle\sum_{a}(pop_{r} \times str_{ar} \times wpx_{ar} \times whx_{ar}) = f(pop_{r}, str_{ar}, wpx_{ar}, whx_{ar})
  \end{equation}


  \subsection{Data sources and descriptives}


  \subsection{The Model of Continuous Change}
  
  To measure the isolated effect of demographic aging on the aggregate labor supply, the continuous change model \citep{horiuchiDecompositionMethodBased2008} was used. This model allows decomposing the difference between two summary measures resulting from the same process into several components, each representing the contribution of underlying factors to the process. The process is a function \(f\), taking as arguments the values of the factors (covariates or independent variables) and returning a summary measure (the dependent variable or variable of interest).
  
  \vspace{0.7em}\par
  \citet{horiuchiDecompositionMethodBased2008} demonstrates that as the factors transition from values \(x_i(t_1)\) to \(x_i(t_2)\) between two times \(t_1\) and \(t_2\), the summary measure changes from \(f(x_i(t_1))\) to \(f(x_i(t_2))\), and the difference between these two measures can be decomposed into additive components \(c_{i}\) representing the contribution of changes within each factor to this difference. This is expressed by equation \eqref{eq:hofr}.
  
  \begin{equation} \label{eq:hofr}
    f(x_i(t_2)) - f(x_i(t_1)) = \displaystyle\sum_{i}c_{i}
  \end{equation}
  
  \vspace{0.7em}\par
  With \(i = 1, 2, ..., n\) being the factors involved. The decomposition is based on the assumption that the change in the covariate \(x_i\) occurs continuously or gradually over a real or hypothetical time dimension \(t_1 \rightarrow t_2\). This provides a reasonable justification for the additivity of the effects of the factors and the elimination of interaction terms even if the process in question is a non-additive function of the covariates \citep[p.~786]{horiuchiDecompositionMethodBased2008}.
  
  \vspace{0.7em}\par
  In other words, the continuous change model (CCM) introduces a decomposition function \(g\) which, using a process function \(f\), transforms two series of covariates, the vectors \(X(t_1)\) and \(X(t_2)\) representing the values of the covariates \(X = (x_1, ..., x_i, ..., x_n)\) at times \(t_1\) and \(t_2\), into a series of contributions or effects, the vector \(C = (c_1, ..., c_i, ..., c_n)\). Thus, we can write:
  
  \begin{equation}\label{eq:gfr}
    C = g(X, f)
  \end{equation}
  
  The series \(X\) and \(C\) having the same lengths, results in a perfect correspondence between the factors and the effects, which can be grouped for analysis purposes. Furthermore, the idea of continuity implies that each element \(c_i\) of the series \(C\) is obtained by integration according to the equation:
  
  \begin{equation}\label{eq:mxfr}
    c_i = \int_{x_i(t_1)}^{x_i(t_2)} \frac{\partial f(t)}{\partial x_i(t)} dx_i(t)
  \end{equation}
  
  In practice, equation \eqref{eq:mxfr} is estimated by numerical integration where \(\int_{x_i(t_1)}^{x_i(t_2)} d(t)\) is approximated by \(\sum_{x_i(t_1)}^{x_i(t_2)} \delta(t)\), with \(x_{it} = x_i(t)\) being the value of a factor \(i\) at a moment \(t\). Adapting the continuous change model to measure equation \eqref{eq:pclab} allowed the total variation in the aggregate labor supply between 1981 and 2016 to be distributed not only among these three components but also by age group and residence status (immigrants and natives). For this article, the decomposition was performed by creating an auxiliary function \(g\) around the R package \citep{Rstat:2018} DemoDecomp \citep{DemoDecomp:2018}. The code and data used are available upon request.
  
  \vspace{0.7em}\par
  The CCM model is a better alternative to the standardization method often used in demography to account for differences in population structure when comparing demographic measures. Indeed, standardization requires a third population to serve as a standard, making it somewhat less robust as different standards can lead to different results. The CCM model solves this problem and allows for direct decomposition without interaction terms between the different factors.
  
  Interaction effects in regression analysis refer to situations where the impact of one variable on a dependent variable is not constant or additive but depends on the values of other variables. In traditional decomposition methods, such as standardization, any discrepancy between the overall change and the sum of the individual effects of the variables is also called an interaction effect. In such cases, the interaction represents an incomplete separation of the contributions of individual covariates to the overall change or difference in a dependent variable.

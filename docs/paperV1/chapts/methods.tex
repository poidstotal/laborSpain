\section{Méthodologies et Données}\label{sec:lab_method}

\subsection{Mesure de l'offre agrégée de travail}
L’objectif principal de cet article est de mesurer l’effet du vieillissement démographique sur l’offre agrégée de travail tout en tenant compte de l'évolution des taux de participation et du nombre d’heures travaillées.
Pour ce faire, l'offre agrégée de travail \(L \) est estimée par la somme pondérée des heures travaillées au cours de l'année par âge \(a \) et par statut de résidence \(r \) (\(whx_{ar} \)), avec \(r = (immigrant, natif)\).
Le facteur de pondération est le produit de l'effectif total de la population (\(pop_{r}\), de la structure de la population (\(str_ar \)) et des taux de participation au travail (\(wpx_{ar}\)) par âge et par statut de résidence.

\begin{equation}\label{eq:pclab}
  L= \displaystyle\sum_{r}\displaystyle\sum_{a}(pop_{r} \times str_{ar} \times wpx_{ar} \times whx_{ar} )=f(pop_{r}, str_{ar}, wpx_{ar}, whx_{ar})
\end{equation}

En exprimant l'offre agrégée de travail total en heures, l’équation \eqref{eq:pclab} produit une mesure plus précise que celle utilisant l'effectif de la population active.
De même, l'utilisation des taux de participation permet une approximation de l'offre agrégée de travail, meilleure que l'emploi total, le total des heures travaillées dans l’économie. 
Ce dernier résulte de l’équilibre sur le marché du travail entre la demande de travail des entreprises et l’offre de travail des ménages. 
En plus, de l'emploi total, l’équation \eqref{eq:pclab} inclut l'offre non absorbée par le marché du travail, estimée environs à 10\% de l'offre annuelle entre 1981 et 2016 (reference Gilbert et al).
Cette offre provient des personnes au chômage ou en emplois à temps partiel, et représente le manque à gagner du sous-emploi.
Toutefois, l’équation \eqref{eq:pclab} se base sur assume que les personnes en chômage ont le désir et la disponibilité de travailler les mêmes heures travaillées per âge des personnes en emploi.

\subsection{Le modèle de changement continue}
Pour mesurer l’effet isolé du vieillissement démographique sur l’offre agrégée de travail, le modèle de changement continu \citep{horiuchiDecompositionMethodBased2008} a été utilisé.
Ce modèle permet de décomposer la différence entre deux mesures récapitulatives issues du même processus en un certain nombre de composantes, chacune représentant la contribution des facteurs sous-jacents au processus.
Le processus est une fonction \(f \), prenant comme arguments les valeurs des facteurs (covariables ou variables indépendantes) et retournant une mesure récapitulative (la variable dépendante ou variable d'intérêt).

\vspace{0.7em}\par
\citet{horiuchiDecompositionMethodBased2008} démontre qu’à mesure que les facteurs passent des valeurs \(x_i(t_1) \) à \(x_i(t_2) \) entre deux temps \(t_1 \) et \(t_2 \), la mesure récapitulative change de \(f(x_i(t_1) )\) à \(f(x_i(t_2) )\) et la différence entre ces deux mesures peut être décomposée en composantes additives \( c_{i} \) représentant la contribution des changements au sein de chaque facteur à cette différence.
Ceci est traduit par l'équation \eqref{eq:hofr}.

\begin{equation} \label{eq:hofr}
  f(x_i(t_2)) - f(x_i(t_1)) = \displaystyle\sum_ {i}c_{i}
\end{equation}

\vspace{0.7em}\par
Avec \( i = 1, 2, . . . , n \) les facteurs en jeu.
La décomposition est basée sur l'hypothèse que le changement de la covariable \(x_i \) se produit de manière continue ou graduelle selon une dimension temporelle réelle ou hypothétique \(t_1 \rightarrow t_2 \).
Il fournit donc une justification raisonnable de l'additivité des effets des facteurs et de l'élimination des termes d'interactions même si le processus en question est une fonction non additive des covariables \citep[p.~786]{horiuchiDecompositionMethodBased2008}.

\vspace{0.7em}\par
En d'autres termes, le modèle MCC introduit une fonction de décomposition \(g \) qui à l'aide d'une fonction de processus \(f \) transforme deux séries de covariables, les vecteurs  \(X(t_1) \) et \(X(t_2)  \) représentant les valeurs des covariables \(X = (x_1,...,x_i...,x_n) \) aux temps \(t_1 \) et \(t_2 \) en une série de contributions ou d'effets, le vecteur \( C = (c_1,...,c_i...,c_n) \).
On peut donc écrire:

\begin{equation}\label{eq:gfr}
  C=g(X,f)
\end{equation}

Les séries \(X \) et \(C \) ayant les mêmes longueurs, il en résulte une correspondance parfaite entre les facteurs et les effets, ces derniers pouvant faire l'objet de différents regroupements pour des fins d'analyse.
De plus, l'idée de continuité implique que chaque élément \(c_i \) de la série \(C \) est obtenu par intégration selon l'équation:

\begin{equation}\label{eq:mxfr}
  c_i=\int_{x_i(t_1)}^{x_i(t_2)} \frac{\partial f(t)}{\partial x_i(t)} dx_i(t)
\end{equation}

Dans la pratique, l'équation \eqref{eq:mxfr} est estimé par une intégration numérique où \( \int_{x_i(t_1)}^{x_i(t_2)} d(t)\) est approximé par \( \sum_{x_i(t_1)}^{x_i(t_2)} \delta(t)\), avec \(x_{it} = x_i(t) \) la valeur d'un facteur \(i \) à un moment \(t \).
L’adaptation du modèle de changement continu à la mesure à l’équation \eqref{eq:pclab} a permis de distribuer la variation totale de l’offre agrégée de travail entre 1981 et 2016, non seulement entre ces trois composantes, mais aussi par groupe d'âge, et statut de résidence (immigrants et natifs). Dans le cadre de cet article, la décomposition à été réalisée en créant une fonction auxiliaire \(g\) autour du package R \citep{Rstat:2018} DemoDecomp \citep{DemoDecomp:2018}. Le code et les données utilisées sont disponibles sur demande.

\vspace{0.7em}\par
Le modèle MCC constitue une meilleure alternative pour la méthode de standardisation souvent utilisée en démographie pour tenir compte des différences de structure de population dans la comparaison de mesures démographiques.
En effet, le fait que la standardisation nécessite une population tierce pour servir de standard la rend quelque peu moins robuste, car différents standards peuvent aboutir à différents résultats.
Le modèle MCC résout ce problème et permet une décomposition directe et sans terme d'interaction entre les différents facteurs.

Les effets d'interaction dans l'analyse de régression se réfèrent à des situations où l'impact d'une variable sur une variable dépendante n'est pas constant ou additif, mais dépend des valeurs d'autres variables. Dans les méthodes de décomposition traditionnelles, telles que la standardisation, toute disparité entre le changement global et la somme des effets individuels des variables est aussi appelée un effet d'interaction. Dans de tels cas, l'interaction représente une séparation incomplète des contributions des covariables individuelles au changement global ou à la différence dans une variable dépendante.

Cela se produit parce que la décomposition était basée sur un changement discret de chaque covariable de la première période à la deuxième tout en maintenant les autres covariables constantes à certains niveaux. En revanche, le MCC repose sur des changements continus des covariables, ce qui rend impossible l'entrée de tout effet d'interaction dans l'équation de décomposition. Avec une compréhension complète du processus de transition continu entre les deux périodes, le passage d'un état à un autre peut être décrit sans effets d'interaction, offrant ainsi une séparation claire des effets des variables individuelles.